
	\section{Determining Size of Nano-islands Through Computer Vision}
	
	Results of applying the computer-vision analysis to the nano-island data archive measure the size distribution of several types of nano-islands formed by annealing.
	
    \subsection{Evaluation Methods for determining Nano-island size}
	A computer-vision method was selected from the research in section [section number]. 
	Nano-island size was evaluated from AFM measurements of the following nano-island samples.
	
	The size is occasionally split between small islands ($20-50 nm$) and larger islands ($>100 nm$).  
	Small features appear in the clear annealed glass samples, 
	so it may be that some small features measured by the AFM are features of the glass and not the gold.
		
	
	\begin{tabular}{c | c || c}
		\hline 
		Clear glass & 0 layers & \\
		\hline
		2 Au & 2 layers & \\
		\hline
		5 Ag & 5 layers & \\
		\hline
		3 Au 3 Ag & 6 layers & \\
		3 Ag 3 Au & 6 layers & not sure if there is a difference\\
		(Au Ag)x3 & 6 layers & \\
		\hline
		5 Ag 1 Au & 6 layers & \\
		5 Au 1 Ag & 6 layers & \\
		6 Ag & 6 layers & \\
		6 Au & 6 layers & \\
		\hline
		5 Ag 5 Au & 10 layers & \\
		\hline
		5 Au 5 Pd & 10 layers & \\
		5 Pd 5 Au & 10 layers & not sure if there is a difference\\
		\hline
		5 Cs Np & unk & pretty sure there's no neptunium \\
		8 Au & 8 layers & \\
		1 Au 3 Ag & 4 layers & \\
		Au 600 & unk & \\
		Others & unk & Need to check AFM log book for file names \\
		\hline 
	\end{tabular}	