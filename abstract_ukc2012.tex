SUMMARY
In-situ impedance measurements during thermolysis monitor the formation of nano-islands from a self-assembled Au-314 multilayer film. 
These measurements provide an insight to the thermolysis process of the polymers and the formation of nano-islands.

I. INTRODUCTION
Gold nano-islands are very sensitive to any plasmon changes on their surface boundary, offering chemical and bio-sensing applications.  
Polymer-mediated self-assembly is a low-cost alternative to evaporation or sputtering for depositing gold on a substrate [1,2]. 
Gold nanoparticles may be functionalized by organic molecules and deposited layer by layer using self-assembly to create layers predictably.
Nano-islands are formed by thermolysis of the organic molecules at high temperatures, and
subsequently annealing the resulting gold film at a temperature between 600°C to 650°C.

II. PROCEDURE
A pre-patterned inter-digitated electrode (IDE) glass substrate was dipped to a solution of Au-314 nano-particles functionalized with organic ligands to form a monolayer. 
Between layers, the electrode was dipped in poly(allyl amine) hydrochloride (PAH), which bonds to an additional layer of functionalized Au-314. Eight layers were assembled by this layer-by-layer method.
The sample impedance was measured in situ during the high temperature heating and cooling by a four-point measurement at 1kHz frequency.

III. RESULTS
Compared to the bare IDE substrate, the self-assembled gold sample shows several distinct
features while heating. The most notable feature is a sharp decrease in impedance at 560°C, reaching less than 100 Ω at 570°C.
Above 620°C the resistance increases sharply.

Figure: Resistance versus temperature is plotted for a multilayer sample (black) and for a bare IDE (blue). 
The arrows indicate the direction of the temperature changes during the measurement.

The sharp decrease in resistance around 560°C suggests the structural integrity of polymers is compromised and the gold nano-particles start forming the conductive percolative network.
The increased resistance at 625°C indicates the formation of nano-islands through the aggregation and the embedding of nano-islands on the glass.
We will present experimental evidences to support our conclusions.

We would like to thank Prof. Shon and his group for the samples and other support.

REFERENCES

Shon, Bull. Korean Chem. Soc. 2010, Vol. 31, No. 2 291.
Shon, Kwon. J. Phys. Chem. C, 2011, 115 (21), pp 10597–10605
