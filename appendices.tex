
	\section{Appendix A: Furnace Control and Temperature Measurement}

	A furnace controller was designed to limit the heating of the sample to improve temperature measurement accuracy.

	Source code for the furnace controller, and version-locked libraries for PID control and MAX31855 Thermocouple reader, are kept at a \url{https://github.com/John-NY/Arduino_Furnace_Controller_PID}.

	\subsection{Furnace Controller block diagram}

	\subsection{Furnace Controller Bill of Materials}

	The furnace temperature controller was programmed on hardware consisting of an Arduino microcontroller and McLaughlin Engineering four channel MAX31855 K-type thermocouple interface board (Arduino ``shield''). 

	A PID implementation was optimized for switch-type relays, that minimizes the number of switch events by limiting switch events to those of 1 second duration or longer.
	Lifespan of a mechanical switch-type relay is measured in switch events, so minimizing the number of switch events will prolong life of the relay and minimize the likelihood of relay failure.
	

		\begin{itemize}

			\item McLaughlin Engineering - MAX6675, MAX31855 breakout
				\subitem Link: \url{http://ryanjmclaughlin.com/wiki}
				\subitem Dimensions: Arduino Shield
				\subitem Chip: MAX31855, TXB0104
				\subitem Voltage: 5V, 3.3V
				\subitem Header: 2 - 1x6 0.1" Pitch, 2 - 1x8 0.1" Pitch
				\subitem Fixturing: Matching holes to Arduino

			\item Omega Type-K Thermocouple
				\subitem Bare Wire 0.010 K-type
				\subitem Mini Connector K-type

			\item Arduino Microcontroller

				\subitem Link: \url{http://arduino.cc}

		\end{itemize}

	\subsection{Furnace Controller Schematic Block Diagram}
	The temperature controller was 

	\subsection{Arduino Code: Furnace Controller PID and Serial Ramp}

