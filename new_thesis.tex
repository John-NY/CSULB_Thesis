\documentclass[12pt,oneside,english]{article}

\usepackage[T1]{fontenc}
\usepackage[latin1]{inputenc}
\usepackage{geometry}
\geometry{verbose,letterpaper,tmargin=1in,bmargin=1in,lmargin=1in,rmargin=1in}
\usepackage{textcomp}
\usepackage{babel}
\setcounter{secnumdepth}{0}
\usepackage{graphicx}
\usepackage{float}
\floatstyle{boxed}
\restylefloat{figure}
\usepackage{longtable}
\usepackage{url}

\newcommand{\BibTeX}{{\sc Bib}\TeX}


\begin{document}

\section{SCOPE}
    This document describes methods for creation and analysis of self-assembled gold nanoislands.
    In-situ impedance measurement has been used to measure apparent changes in state of the gold nano-islands.
    Here we find and reduce sources of experimental error to demonstrate a better in-situ measurement.
    The in-situ measurement includes use of a custom oven controller / thermocouple reader created from an Arduino microcontroller.
    The analysis of resulting nano-island samples is derived from Atomic Force Microscope images -- the sparse islands benefit from a ``blob finder'' approach using computer vision libraries and analysis that measures cross-sections.



\section{Testbed Hardware}
    \subsection{Self-Assembling Multilayer Film Wet Chemistry Station}
        A wet chemistry station was set up for the creation of multilayer nanoisland slides.
        

    \subsection{In-situ impedance measurement thermolysis and annealing station}
    		A sketch was written in NI LabView to monitor temperatures and Impedance measurements.  A lab furnace used for thermolysis and annealing was customized with temperature control and measurement electronics.  
    		
    	\subsection{Atomic Force Microscope and Scanning Electron Microscope}
    	
    	
    	\subsection{Analysis and Image Processing Workstation}

\section{Introduction}
    Copy background research
    	
\end{document}
